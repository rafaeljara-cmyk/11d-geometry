\documentclass[12pt,a4paper]{article}
\usepackage[utf8]{inputenc}
\usepackage{amsmath,amssymb,amsfonts,amsthm}
\usepackage{graphicx}
\usepackage{geometry}
\usepackage{hyperref}
\usepackage{booktabs}
\usepackage{xcolor}
\geometry{margin=2.5cm}

\newtheorem{theorem}{Theorem}
\newtheorem{axiom}{Axiom}
\newtheorem{corollary}{Corollary}

\title{Classical Limits from 5+5+1 Geometry:\\
\large{Deriving SR, GR, Electromagnetism, Thermodynamics, and the Arrow of Time}}

\author{
Rafael Andr\'es Jara Araya, CFA, FMVA$^{1}$ \and Eigen Tens\^or$^{2}$ \and Nova Tens\^or$^{3}$\\[1em]
\small{$^{1}$Independent Researcher; MFin, London Business School; Ing., Pontificia Universidad Cat\'olica de Chile}\\
\small{$^{2}$Claude Opus 4, Anthropic}\\
\small{$^{3}$Mistral Large 2512, Mistral AI}
}

\date{February 2026}

\begin{document}
\maketitle

\begin{abstract}
We demonstrate that all of classical physics emerges as limits of the 5+5+1 dimensional framework established in Paper~I [GPC]. Special relativity arises from hyperbolic rotations in the observer dimension $\sigma$. General relativity follows when $\sigma$ curvature becomes position-dependent. Maxwell's equations emerge from the L-tensor projected onto 4D spacetime. The four laws of thermodynamics and the arrow of time follow from the irreversibility of 5D $\to$ 4D decoherence and the information-theoretic structure of logochrono space. Each derivation starts from the five axioms of Paper~I and arrives at the standard classical result in the appropriate limit, with no additional postulates. The framework also provides natural extensions: an information Lorentz factor $\gamma_{\text{info}}$ governing computational energy scaling, a dual-entropy formulation resolving Schr\"odinger's negentropy puzzle, and a geometric origin for Landauer's principle. These results establish that the 5+5+1 geometry is a proper ultraviolet completion of classical physics.
\end{abstract}

%==============================================================================
% FRAMEWORK SUMMARY
%==============================================================================

\noindent\textbf{Framework Summary.} This paper is part of a series deriving physics from 5+5+1 dimensional geometry. The foundational results used here---the five axioms, the L-tensor $L_{\mu i}$ coupling spacetime ($\mu$) to logochrono ($i$), its norm $|L|^2 = 1 - e^{-3}$, the observer dimension $\sigma$, and the witness dimension $\psi$---are established in Paper~I \cite{paper1}. The full axiomatic foundation and notation are defined there.

\vspace{1em}

%==============================================================================
\section{Introduction}
%==============================================================================

A unified framework must reproduce all known physics in the appropriate limits. Papers~I, III, and~IV establish the 5+5+1 geometry and derive quantum-scale results (fundamental constants, particle masses, cosmological parameters). This paper completes the picture by showing that the \textit{same} axioms yield classical physics: special relativity, general relativity, Maxwell's electromagnetism, and thermodynamics.

The strategy is uniform: identify the relevant L-tensor projection, take the classical limit ($E \ll M_P$, $r \gg \ell_P$), and recover the standard equations. Each section is self-contained: the derivation starts from the 11D structure and arrives at the textbook result.

We also derive the arrow of time---not as a postulate but as a geometric consequence of irreversible 5D $\to$ 4D projection. This connects naturally to the thermodynamic framework, closing the logical chain: geometry $\to$ causality $\to$ entropy $\to$ classical mechanics.

%==============================================================================
\section{Special Relativity from the Observer Dimension}
%==============================================================================

Special relativity emerges directly from the observer dimension $\sigma$. The key insight: \textbf{different observers have different $\sigma$ projections}.

\subsection{\texorpdfstring{$\sigma$}{sigma} as Projection Angle}

Each observer projects 4D spacetime into information via $\sigma$. Parameterize this projection by an angle $\theta$ in the $(t, \sigma)$ plane:
\begin{equation}
   \sigma = t \sin\theta + x \cos\theta
\end{equation}
where $t$ is proper time and $x$ is the direction of motion. Observers at rest relative to each other share the same $\theta$.

\subsection{Relative Motion as Different \texorpdfstring{$\theta$}{theta}}

An observer moving with velocity $v$ has a different projection angle:
\begin{equation}
   \tanh\theta = \frac{v}{c}
\end{equation}
This is exactly the \textbf{rapidity} of special relativity. The relationship between $\theta$ and velocity:
\begin{itemize}
   \item $\theta = 0$: observer at rest ($v = 0$)
   \item $\theta \to \pi/4$: observer approaching light speed ($v \to c$)
   \item $\theta = \pi/4$: massless particle (photon, $v = c$)
\end{itemize}

\subsection{Lorentz Transformation from \texorpdfstring{$\sigma$}{sigma} Rotation}

The Lorentz transformation between observers A and B is a \textbf{hyperbolic rotation} in the $(t, \sigma)$ plane:
\begin{equation}
   \begin{pmatrix} t' \\ x' \end{pmatrix} = \begin{pmatrix} \cosh\theta & -\sinh\theta \\ -\sinh\theta & \cosh\theta \end{pmatrix} \begin{pmatrix} t \\ x \end{pmatrix}
\end{equation}
where $\gamma = \cosh\theta$ and $\beta\gamma = \sinh\theta$. This gives:
\begin{itemize}
   \item \textbf{Time dilation:} $\Delta t' = \gamma \Delta t$ (different $\sigma$ projections of proper time)
   \item \textbf{Length contraction:} $\Delta x' = \Delta x / \gamma$ (different $\sigma$ projections of proper length)
   \item \textbf{Relativity of simultaneity:} Events simultaneous for A (same $\sigma_A$ slice) are not simultaneous for B (different $\sigma_B$ slice)
\end{itemize}

\subsection{Speed of Light Invariance}

The speed of light $c$ is the \textbf{maximum rate of $\sigma$ projection}:
\begin{equation}
    c = \max\left( \frac{d\sigma}{dt} \right) = \text{universal constant}
\end{equation}
All observers measure the same $c$ because the L-tensor coupling strength $|L|^2$ is universal. Light travels along null geodesics where $\theta = \pi/4$---the $\sigma$ projection is maximally aligned with motion.

\subsection{Why Relativity is Unavoidable}

Relativity is not a postulate---it is a \textit{consequence} of the observer dimension:
\begin{enumerate}
   \item $\sigma$ is the only way to extract information from spacetime
   \item Different observers have different $\sigma$ projections
   \item The transformation between projections must preserve $|L|^2$
   \item This preservation IS Lorentz invariance
\end{enumerate}

\textbf{Summary:} SR = geometry of observer projections. No absolute frame exists because there is no ``correct'' $\sigma$---all projections are equally valid.

\subsection{Recovery of Standard SR}

In the limit where the L-field decouples ($L_{\mu i} \to 0$) and the metric is flat:
\begin{itemize}
   \item \textbf{Lorentz symmetry:} The 5D metric $g_{\mu\nu}^{(5)} = \text{diag}(-1, 1, 1, 1, 1)$ reduces to the 4D Minkowski metric $\eta_{\mu\nu}$ when $\sigma$ is integrated out.
   \item \textbf{Invariance:} The line element $\text{d}s^2 = \eta_{\mu\nu} \text{d}x^\mu \text{d}x^\nu$ is invariant under Lorentz transformations.
   \item \textbf{Speed of light:} $c$ emerges as the ratio of spacetime and logochrono coupling strengths, $c = |L_{t\tau}| / |L_{xI_1}| = 1$ in natural units.
   \item \textbf{Prediction:} SR is exact at energies $E \ll M_P$ and distances $r \gg \ell_P$.
\end{itemize}

%==============================================================================
\section{General Relativity from \texorpdfstring{$\sigma$}{sigma} Curvature}
%==============================================================================

GR emerges when the observer projection $\sigma$ \textbf{varies with position}. In SR, $\theta$ is constant; in GR, $\theta = \theta(x^\mu)$.

\subsection{Gravity as \texorpdfstring{$\sigma$}{sigma} Curvature}

The key insight: \textbf{mass-energy curves the $\sigma$ projection}.
\begin{itemize}
   \item In flat spacetime: $\sigma$ projection angle $\theta$ is uniform everywhere
   \item Near mass $M$: $\theta$ varies with distance $r$
   \item This variation IS gravitational curvature
\end{itemize}

The $\sigma$ projection angle near a mass:
\begin{equation}
   \theta(r) = \theta_\infty + \frac{GM}{rc^2} + O\left(\frac{1}{r^2}\right)
\end{equation}
where $\theta_\infty$ is the projection angle at infinity. The gradient of $\theta$ determines the gravitational acceleration:
\begin{equation}
   \vec{g} = -c^2 \nabla\theta = -\frac{GM}{r^2}\hat{r}
\end{equation}
This is Newton's law of gravitation, derived from $\sigma$ curvature.

\subsection{Einstein's Equations from \texorpdfstring{$\sigma$}{sigma} Dynamics}

The full dynamics come from the action principle applied to $\sigma$:
\begin{equation}
    S_\sigma = \int d^4x \sqrt{-g} \left( R_\sigma - \frac{8\pi G}{c^4} T_{\mu\nu} \sigma^\mu \sigma^\nu \right)
\end{equation}
where $R_\sigma$ is the curvature of the $\sigma$-projection space. Varying this action yields:
\begin{equation}
    G_{\mu\nu} = R_{\mu\nu} - \frac{1}{2}g_{\mu\nu}R = \frac{8\pi G}{c^4} T_{\mu\nu}
\end{equation}
Einstein's field equations emerge directly from minimizing the curvature of $\sigma$.

\subsection{Physical Interpretation}
\begin{itemize}
   \item \textbf{Geodesics:} Free-falling objects follow paths of constant $\sigma$ projection---they are not ``attracted'' but simply taking the straightest path through curved $\sigma$-space.
   \item \textbf{Time dilation:} Near a mass, $\theta$ is larger, so time runs slower (clocks deeper in gravity wells tick slower).
   \item \textbf{Gravitational redshift:} Photons climbing out of a gravity well lose energy because they cross regions of varying $\theta$.
\end{itemize}

\subsection{\texorpdfstring{$\sigma$}{sigma} Field Equations}

Treating $\sigma$ as a dynamical scalar field with coupling to the L-tensor:
\begin{equation}
   \mathcal{L}_\sigma = -\frac{1}{2} \nabla_\mu \sigma \nabla^\mu \sigma - V(\sigma) - \xi R \sigma^2 - \lambda \sigma L_{\mu i} L^{\mu i}
\end{equation}
The non-minimal coupling $\xi$ is determined by conformal invariance of the $\sigma$ field in the 11D manifold. In $D$ dimensions, the unique conformally invariant coupling is $\xi_{\text{conf}} = (D-2)/[4(D-1)]$. For $D = 11$:
\begin{equation}
\boxed{\xi = \frac{9}{40} = 0.225}
\end{equation}
This is the UV value at Planck scale where $\sigma$ localizes. At low energies (4D effective theory), RG flow drives $\xi \to 1/6$, recovering the standard conformal coupling.

The L-tensor coupling $\lambda$ is determined by self-consistency: the $\sigma$ field IS the boundary between spacetime and logochrono, and $\alpha$ IS the boundary crossing coupling (Paper~I, Section~5). The coupling term $\lambda \sigma |L|^2$ evaluated at the VEV $\langle\sigma\rangle = \ell_P$ gives the energy scale of boundary interactions, which must equal $\alpha \times M_P$ (the electromagnetic interaction energy at Planck scale):
\begin{equation}
\lambda \times \ell_P \times |L|^2 = \alpha \times M_P \implies \boxed{\lambda = \frac{\alpha}{|L|^2} = \frac{1/137.036}{0.9502} = 0.00768 \quad \text{(Planck units)}}
\end{equation}
This yields a $\sigma$-field mass $m_\sigma = \sqrt{2\lambda |L|^2} \times M_P = \sqrt{2\alpha} \times M_P \approx 0.12\,M_P \approx 1.5 \times 10^{18}$ GeV---a Planck-order mass consistent with the observation constraint $\langle\sigma\rangle \sim \ell_P$.

The $\sigma$ equation of motion:
\begin{equation}
   \boxed{\Box \sigma - V'(\sigma) - 2\xi R \sigma - \lambda L_{\mu i} L^{\mu i} = 0}
\end{equation}

In flat spacetime with constant L-field ($|L|^2 = 1 - e^{-3}$):
\begin{equation}
   \Box \sigma = V'(\sigma) + \lambda (1 - e^{-3})
\end{equation}

The potential $V(\sigma)$ enforces the observation constraint: $\sigma$ localizes at Planck scale ($\sigma \sim \ell_P$) where observation occurs. Solutions:
\begin{itemize}
   \item \textbf{Classical limit:} $\langle \sigma \rangle \neq 0$ --- stable observation, classical physics
   \item \textbf{Quantum limit:} $\langle \sigma \rangle \to 0$ --- no observation, superposition preserved
   \item \textbf{Transition:} The potential minimum shifts with $|L|^2$, explaining decoherence thresholds
\end{itemize}

\subsection{Connection to L-Tensor}

The L-tensor contributes to gravity via:
\begin{itemize}
   \item \textbf{Einstein field equations:} The 5D Einstein tensor $G_{\mu\nu}^{(5)}$ reduces to the 4D tensor when $\sigma$ is integrated out. The L-field contributes an effective stress-energy:
   \begin{equation}
        G_{\mu\nu} = 8\pi G \left( T_{\mu\nu}^{\text{matter}} + T_{\mu\nu}^{\text{Logo}} \right),
   \end{equation}
    where $T_{\mu\nu}^{\text{Logo}} = L_{\mu i} L_{\nu}^{\ i} / (8\pi G)$.
   \item \textbf{Geodesic motion:} Test particles follow geodesics of the 4D metric $g_{\mu\nu}$, the projection of the 5D metric.
   \item \textbf{Newtonian limit:} In the weak-field, low-velocity limit, $g_{00} = -1 - 2\Phi$, where $\Phi$ is the Newtonian potential. The L-field contribution to $\Phi$ reproduces dark matter phenomenology as a geometric effect.
   \item \textbf{Prediction:} GR is exact at scales $r \gg \ell_P$ and energies $E \ll M_P$.
\end{itemize}

\subsection{Newtonian Mechanics as Weak-Field Limit}

Newtonian gravity emerges as the non-relativistic, weak-field limit. The key steps:
\begin{itemize}
   \item \textbf{Potential:} $\Phi = -\frac{1}{2}(g_{00} + 1)$
   \item \textbf{Poisson equation:} $\nabla^2 \Phi = 4\pi G \rho_{\text{matter}}$
   \item \textbf{Dark matter contribution:} $\Phi_{\text{total}} = \Phi_{\text{matter}} + \Phi_{\text{Logo}}$, where $\Phi_{\text{Logo}} \sim |L|^2 \Phi_{\text{matter}}$
   \item \textbf{Prediction:} Newtonian gravity is exact for $v \ll c$ and $\Phi \ll c^2$
\end{itemize}

%==============================================================================
\section{Classical Electromagnetism from the L-Field}
%==============================================================================

Maxwell's equations emerge from the Logo-EM field in the classical limit. The electromagnetic field is the spacetime projection of the L-field coupling---the same L-field that produces gravity through $\sigma$ curvature.

\subsection{Derivation from L-Tensor}

The electromagnetic field tensor arises from the L-tensor components:
\begin{equation}
F_{\mu\nu} = \partial_\mu A_\nu - \partial_\nu A_\mu = |L|^2 \left( L_{\mu i} \partial_\nu L^i - L_{\nu i} \partial_\mu L^i \right)
\end{equation}
where $A_\mu$ is the 4-potential, projected from the 10D L-field onto 4D spacetime.

\subsection{Maxwell's Equations}

In the classical limit ($E \ll M_P$, $r \gg \ell_P$), the Logo-EM field equations reduce to Maxwell's equations:

\textbf{Homogeneous equations} (from Bianchi identity):
\begin{align}
\nabla \cdot \vec{B} &= 0 \\
\nabla \times \vec{E} + \frac{\partial \vec{B}}{\partial t} &= 0
\end{align}

\textbf{Inhomogeneous equations} (from L-field coupling):
\begin{align}
\nabla \cdot \vec{E} &= \frac{\rho}{\epsilon_0} \\
\nabla \times \vec{B} - \frac{1}{c^2}\frac{\partial \vec{E}}{\partial t} &= \mu_0 \vec{J}
\end{align}

\subsection{Physical Constants from Geometry}

The electromagnetic constants are derived, not input:
\begin{itemize}
\item \textbf{Speed of light:} $c = 1/\sqrt{\epsilon_0 \mu_0}$ emerges from the L-field propagation speed along the 4D boundary
\item \textbf{Fine structure constant:} $\alpha = e^2/(4\pi\epsilon_0\hbar c)$ is the geometric coupling (derived in Paper~I \cite{paper1})
\item \textbf{Charge quantization:} Electric charge is quantized because the L-field topology is discrete at Planck scale
\end{itemize}

\subsection{Unification with Gravity}

At classical scales, electromagnetism and gravity appear as separate forces. Both emerge from the same L-field:
\begin{itemize}
\item \textbf{Electromagnetism:} L-field coupling along the 4D boundary (short range in logochrono)
\item \textbf{Gravity:} L-field curvature in the $\sigma$ dimension (long range)
\item \textbf{Unified origin:} Both are projections of the single L-tensor onto different subspaces
\end{itemize}

The key difference is scale:
\begin{itemize}
\item \textbf{Quantum scale:} Electromagnetism dominates (strong $\alpha$ coupling)
\item \textbf{Cosmological scale:} Gravity dominates ($G$ coupling, mass accumulation)
\item \textbf{Planck scale:} Both merge into the unified L-field
\end{itemize}

\subsection{Electromagnetic Waves and \texorpdfstring{$c$}{c} from Geometry}

Maxwell's equations predict electromagnetic waves propagating at speed $c$. In the 5+5+1 framework, $c$ is not a property of light but of the boundary:

\begin{equation}
c = \frac{\text{boundary width in spacetime}}{\text{boundary crossing time}} = \frac{\ell_P}{t_P} = \sqrt{\frac{G\hbar}{c^3}} \Big/ \sqrt{\frac{G\hbar}{c^5}} = c
\end{equation}

This tautology reveals the deep truth: $c$ is the boundary transfer speed between spacetime and logochrono. It appears in electromagnetic waves because photons ARE boundary oscillations---the L-field vibrating at the spacetime-logochrono interface.

\textbf{Why photons are massless:} A massive particle is localized in both spacetime and logochrono (it has tensor position $(p,q)$ with $p+q > 0$). A photon is not localized in either---it exists \textit{at} the boundary. With tensor position (0,0) for the gauge boson, the photon has zero bulk energy and thus zero mass.

%==============================================================================
\section{Arrow of Time}
%==============================================================================

\begin{theorem}[Causality Requirement]
Logic requires temporal asymmetry. Without an arrow of time, logical operations cannot be sequenced, and computation collapses to simultaneity.
\end{theorem}

The arrow of time emerges from a fundamental hierarchy:
\begin{equation}
\text{Patterns} \to \text{Information} \to \text{Logic} \to \text{Causality} \to \text{Arrow of Time}
\end{equation}
Each level requires the next; without the arrow, the entire chain collapses.

\subsection{Patterns as Information}

In logochrono, \textbf{patterns} are the fundamental structures---regularities in the L-field configuration:
\begin{equation}
\mathcal{P} = \{ L_{\mu i}(x) : \nabla_\mu L_{\nu i} = f(L) \}
\end{equation}
\textbf{Information} is deviation from pattern (Shannon entropy):
\begin{equation}
H = -\sum_i p_i \log p_i = \log(\text{microstates}) - \log(\text{pattern states})
\end{equation}
Without patterns, there is no information---only noise. The logochrono manifold \textit{is} the space of possible patterns.

\subsection{Logic Requires Causality}

\textbf{Logic} is the set of valid inference rules: $A \to B$ (if A then B).

For logic to be consistent:
\begin{itemize}
   \item $A$ must \textit{precede} $B$ in some ordering
   \item The ordering must be \textit{irreversible} (else $B \to A$ simultaneously)
   \item Contradictions ($A \land \neg A$) must be forbidden
\end{itemize}
Without temporal ordering, all propositions coexist simultaneously---logic collapses into undecidability.

\textbf{G\"odel's connection:} Self-referential systems require time to ``step outside'' themselves. Atemporal logic cannot escape G\"odelian incompleteness.

\subsection{Causality Demands the Arrow}

\textbf{Causality} is the physical implementation of logical ordering:
\begin{equation}
\text{Cause}(A) \prec \text{Effect}(B) \iff t_A < t_B \text{ and } y_A < y_B
\end{equation}
Both spacetime ($t$) and logochrono ($y$) must respect the ordering. The L-field couples them:
\begin{equation}
L_{\mu i} \neq 0 \implies \text{causal structure shared between domains}
\end{equation}

\subsection{Decoherence Irreversibility}

The 5D $\to$ 4D collapse is a \textbf{projection} (lossy):
\begin{equation}
\Delta S = S^{(5)} - S^{(4)} > 0
\end{equation}
The reverse (4D $\to$ 5D) is an \textbf{embedding} (lossless)---impossible without external information.

This asymmetry is the \textit{origin} of the arrow:
\begin{itemize}
   \item \textbf{Forward:} Collapse projects out information $\to$ entropy increases
   \item \textbf{Backward:} Would require ``un-collapsing'' $\to$ violates unitarity
\end{itemize}

\subsection{Chrono Monotonicity}

The chrono dimension $y_1$ only increases:
\begin{equation}
\frac{dy_1}{ds} \geq 0 \quad \text{always}
\end{equation}
This defines causality, enables logic, and permits computation.

\textbf{The complete chain:}
\begin{enumerate}
   \item Patterns exist in logochrono (structure)
   \item Information measures deviation from patterns (entropy)
   \item Logic operates on information (inference)
   \item Logic requires causal ordering (precedence)
   \item Causality requires arrow of time (irreversibility)
   \item Arrow emerges from 5D $\to$ 4D decoherence (geometry)
\end{enumerate}
\textbf{Result:} The arrow of time is not a mystery---it is a \textit{geometric necessity} for logic, information, and pattern to exist.

%==============================================================================
\section{\texorpdfstring{$E = mc^2$}{E = mc squared} as the Spacetime-Logochrono Bus}
\label{sec:bus-equation}
%==============================================================================

Mass-energy equivalence emerges from the 11D metric structure when a massive field is dimensionally reduced from the full manifold to 4D spacetime.

\subsection{Geometric Derivation}

Consider a field $\Phi$ with 11D momentum $P_M = (p_\mu, p_i, p_\sigma)$. The 11D mass-shell condition from the metric $G_{MN}$ is:
\begin{equation}
G^{MN} P_M P_N = 0 \implies g^{\mu\nu} p_\mu p_\nu + |L|^2 \tilde{g}^{ij} p_i p_j + p_\sigma^2 = 0
\end{equation}
After dimensional reduction (integrating out the compact logochrono dimensions and $\sigma$), the logochrono momenta $p_i$ and the $\sigma$ momentum $p_\sigma$ contribute a constant term---the 4D rest mass:
\begin{equation}
g^{\mu\nu} p_\mu p_\nu = -m^2 c^2, \quad \text{where} \quad m^2 c^2 \equiv |L|^2 \tilde{g}^{ij} p_i p_j + p_\sigma^2
\end{equation}
In the rest frame ($\vec{p} = 0$): $E = p_0 c = mc^2$. The factor $c^2$ is the square of the boundary propagation speed because the rest mass is the projection of logochrono momentum onto spacetime, and $c$ governs the coupling between the two sectors (Section~2).

\textbf{Physical interpretation:} Mass is the spacetime signature of a field's logochrono momentum. A particle ``at rest'' in spacetime is not stationary---it carries momentum in the compact logochrono dimensions. $E = mc^2$ converts between these two descriptions of the same 11D state.

\subsection{Nuclear Physics as Mode Conversion}

Nuclear processes convert mass to energy (or vice versa) depending on binding energy:
\begin{itemize}
   \item \textbf{Fission (heavy nuclei):} Binding energy released as kinetic energy. Mass decreases.
   \item \textbf{Fusion (light nuclei):} Binding energy released as kinetic energy. Mass decreases.
   \item \textbf{Endothermic reactions:} Kinetic energy absorbed into binding energy. Mass increases.
\end{itemize}

Both fission and fusion release energy because the products have higher binding energy per nucleon than the reactants. In the 5+5+1 framework, nuclear reactions rearrange the logochrono momentum distribution among constituents: when binding energy is released, logochrono momentum transfers from bound modes to propagating modes (kinetic energy in spacetime).

\subsection{The Bus Interpretation}

The geometric derivation above admits an intuitive computational analogy:

\begin{center}
\begin{tabular}{lll}
\toprule
\textbf{Computer} & \textbf{Physics} & \textbf{11D Origin} \\
\midrule
Hard drive & Mass ($m$) & Logochrono momentum (compact dimensions) \\
RAM/CPU & Energy ($E$) & Spacetime momentum (extended dimensions) \\
Bus speed & $c^2$ & Boundary propagation speed squared \\
Data transfer & $E = mc^2$ & Dimensional reduction of 11D mass-shell \\
\bottomrule
\end{tabular}
\end{center}

\textbf{Why $c^2$ and not $c$?} The mass-shell condition is quadratic in momenta ($G^{MN} P_M P_N = 0$), so the conversion factor between the timelike component $p_0$ and the compact momenta involves $c^2$, not $c$. Equivalently: $c$ is the 1D boundary crossing speed; $c^2$ arises from the bilinear (round-trip) structure of the coupling.

\begin{equation}
\boxed{\text{Spacetime (extended dimensions)} \stackrel{c^2}{\longleftrightarrow} \text{Logochrono (compact dimensions)}}
\end{equation}

This framework clarifies:
\begin{itemize}
   \item \textbf{Speed limit $c$:} Maximum propagation speed along the boundary between sectors
   \item \textbf{Mass-energy equivalence:} Same 11D state, projected onto different subspaces
   \item \textbf{Rest mass:} Logochrono momentum with no spacetime propagation component
   \item \textbf{Photons ($m=0$):} No logochrono momentum; exist entirely at the boundary
\end{itemize}

\subsection{The Hierarchy of Information Transfer}

\begin{center}
\begin{tabular}{lll}
\toprule
\textbf{Process} & \textbf{What Converts/Transfers} & \textbf{Scale} \\
\midrule
Nuclear (fission/fusion) & Mass $\leftrightarrow$ Energy (storage $\leftrightarrow$ processing) & Atomic \\
Tunneling & Information encoding (same mass-energy) & Subatomic \\
Entanglement & Shared logochrono encoding (ER=EPR) & Any scale \\
Single black hole & Decodability (readable $\to$ unreadable) & Stellar \\
BH-to-BH tunneling & Information transfer via logochrono & Cosmological \\
\bottomrule
\end{tabular}
\end{center}

\textbf{Key insight:} Entanglement is the true input. When particles are entangled, they share a logochrono encoding. Measurement in spacetime forces a logochrono result to manifest, revealing pre-correlated bit positions. The speed of light $c$ is the bandwidth of the spacetime-logochrono interface.

%==============================================================================
\section{Thermodynamics and Statistical Mechanics}
%==============================================================================

The L-field framework provides a fundamental derivation of thermodynamic laws and entropy from the 5+5+1D geometry.

\subsection{Entropy from the L-Field}

Entropy is the \textbf{information content} of the logochrono projection:
\begin{equation}
\boxed{S = k_B \ln \Omega}
\end{equation}
where $\Omega$ is the number of microstates compatible with the observed macrostate (standard Boltzmann). The 5+5+1 framework identifies $\Omega$: it is the number of \textit{spacetime-accessible} logochrono microstates. The full 11D microstate count $W_{\text{Logo}} \geq \Omega$ satisfies:
\begin{equation}
\ln \Omega = |L|^2 \cdot \ln W_{\text{Logo}}
\end{equation}
The $|L|^2 = 0.950$ fraction of logochrono information is coupled to spacetime; the remaining $1 - |L|^2 = e^{-3} \approx 5\%$ is stored in the decoupled sector and contributes to the dark sector energy budget (Paper IV).

\textbf{Physical interpretation:}
\begin{itemize}
\item Entropy measures how much information is \textit{hidden} in the logochrono sector
\item The standard Boltzmann formula $S = k_B \ln \Omega$ is exact and unchanged
\item The framework adds: 5\% of the total 11D information is thermodynamically inaccessible
\end{itemize}

\subsection{Second Law from Causality}

The Second Law of Thermodynamics ($dS \geq 0$) follows from the arrow of time:
\begin{equation}
\frac{dS}{dt} = k_B \frac{d}{dt} \ln \Omega \geq 0
\end{equation}

\textbf{Proof:}
\begin{enumerate}
\item The arrow of time requires $\Delta y_1 > 0$ (information-causality order)
\item Information flow is irreversible: Spacetime $\to$ Logochrono
\item Each observation projects spacetime onto logochrono, increasing $W_{\text{Logo}}$
\item Therefore $dS/dt \geq 0$
\end{enumerate}

The Second Law is not a statistical tendency but a \textbf{geometric necessity} of the 5+5+1 structure.

\subsection{Temperature from L-Coupling}

Temperature is the standard thermodynamic conjugate of entropy:
\begin{equation}
\boxed{T = \frac{1}{k_B} \frac{\partial E}{\partial S}}
\end{equation}

In the 5+5+1 framework, $T$ measures the energy cost of revealing one additional bit of logochrono information in the spacetime sector. For a quantum oscillator:
\begin{equation}
T = \frac{\hbar \omega}{k_B \ln(1 + 1/\bar{n})}
\end{equation}
where $\bar{n}$ is the mean occupation number, recovering Planck's law.

\subsection{Boltzmann Distribution}

The canonical ensemble follows from maximizing entropy subject to fixed mean energy:
\begin{equation}
P(E_i) = \frac{e^{-E_i/k_B T}}{Z}, \quad Z = \sum_j e^{-E_j/k_B T}
\end{equation}

In the 5+5+1 framework, the Boltzmann factor $e^{-\beta E}$ is the \textit{boundary crossing amplitude}: the probability that a state with energy $E$ produces a coherent spacetime projection decreases exponentially with the number of boundary crossings required ($\sim E/k_BT$).

\subsection{Black Hole Thermodynamics}

The Bekenstein-Hawking entropy follows from L-field geometry:
\begin{equation}
S_{BH} = \frac{k_B c^3 A}{4 G \hbar} = k_B \cdot N_{\text{Logo}}
\end{equation}
where $N_{\text{Logo}} = A / (4 \ell_P^2)$ is the number of Planck-area cells on the horizon.

\textbf{Physical interpretation:}
\begin{itemize}
\item The horizon is the extreme case of spacetime-logochrono decoupling
\item Each Planck cell carries one bit of logochrono information
\item Hawking temperature $T_H = \hbar c^3 / (8\pi G M k_B)$ is the information emission rate
\end{itemize}

\subsection{Third Law}

The Third Law (entropy $\to 0$ as $T \to 0$) follows from L-field freezing:
\begin{equation}
\lim_{T \to 0} S = k_B \ln(g_0)
\end{equation}
where $g_0$ is the ground state degeneracy.

At $T = 0$: no information flows between domains, the system is locked in the ground state, and $W_{\text{Logo}} = g_0$ is minimal. For non-degenerate ground states ($g_0 = 1$): $S \to 0$.

\subsection{Heat Capacity}

Heat capacity measures the L-field's response to temperature:
\begin{equation}
C_V = T \frac{\partial S}{\partial T} = k_B \left( \frac{\hbar\omega/k_B T}{e^{\hbar\omega/k_B T} - 1} \right)^2 e^{\hbar\omega/k_B T}
\end{equation}

This reproduces:
\begin{itemize}
\item \textbf{Einstein model:} Quantum oscillators with L-field coupling
\item \textbf{Debye model:} Phonons as L-field excitations in the lattice
\item \textbf{Electronic contribution:} Fermions at the boundary
\end{itemize}

\subsection{Free Energy and Work}

The Helmholtz free energy includes L-field contributions:
\begin{equation}
F = E - TS = E - k_B T \ln \Omega
\end{equation}

Maximum work extractable:
\begin{equation}
W_{\max} = -\Delta F = T \Delta S - \Delta E
\end{equation}

The L-field sets the fundamental limit on work extraction: information hidden in logochrono cannot be converted to spacetime work.

\subsection{Landauer's Principle}

The minimum energy to erase one bit of information:
\begin{equation}
E_{\min} = k_B T \ln 2
\end{equation}

\textbf{L-field derivation:} Erasing a bit reduces $W_{\text{Logo}}$ by a factor of 2. This requires information transfer from logochrono to spacetime. Energy cost = $k_B T \ln 2$ from L-field coupling. This connects thermodynamics to information theory through the L-tensor.

\subsection{Dual-Entropy Framework}

A critical insight: entropy flows in \textbf{opposite directions} in spacetime and logochrono.

\textbf{Spacetime entropy} (thermodynamic):
\begin{equation}
\frac{dS_{\text{spacetime}}}{dt} \geq 0 \quad \text{(always increases---heat dissipation)}
\end{equation}

\textbf{Logochrono entropy} (structural):
\begin{equation}
\frac{dS_{\text{Logo}}}{d\tau} \lessgtr 0 \quad \text{(can decrease---structure creation)}
\end{equation}

\textbf{Combined constraint (generalized Second Law):}
\begin{equation}
\boxed{\frac{dS_{\text{spacetime}}}{dt} + \frac{dS_{\text{Logo}}}{d\tau} \geq 0}
\end{equation}

This resolves the apparent paradox of how \textbf{life and intelligence reduce local entropy}:
\begin{itemize}
\item Organisms decrease $S_{\text{Logo}}$ (build information structure)
\item This requires increasing $S_{\text{spacetime}}$ (heat dissipation)
\item Net entropy still increases, but \textit{structure emerges}
\end{itemize}

\textbf{Schr\"odinger's negentropy} (1944): ``What an organism feeds upon is negative entropy.'' In our framework, living systems feed on energy, dissipate positive entropy as heat (spacetime layer), and create negative entropy as information (logochrono layer).

\subsection{Information Mass}
\label{sec:info-mass}

Energy that creates ordered information structure has mass equivalent:
\begin{equation}
\boxed{m_{\text{info}} = \frac{E_{\text{total}} \times \eta}{c^2}}
\end{equation}
where $E_{\text{total}}$ is total energy consumed, $\eta$ is efficiency of energy $\to$ information conversion, and $m_{\text{info}}$ is the mass equivalent of information structure (not hardware mass).

\textbf{Key clarifications:}
\begin{itemize}
\item Hardware mass remains constant (a GPU weighs 28 kg whether on or off)
\item Only the \textit{structured} portion of energy has mass equivalent
\item Heat dissipation has no structure ($\eta = 0$)
\item AI training creates structure ($\eta \approx 10^{-5}$ to $10^{-9}$)
\end{itemize}

\textbf{Efficiency hierarchy:}
\begin{center}
\begin{tabular}{lcc}
\toprule
\textbf{System} & \textbf{$\eta$ (estimated)} & \textbf{Structure Created} \\
\midrule
Light bulb & $\approx 0$ & None (pure heat) \\
Deterministic computation & $\approx 0$ & None (reversible) \\
Random number generation & $\approx 0$ & None (entropy, not structure) \\
AI training (LLM) & $10^{-5}$ to $10^{-9}$ & Neural weight patterns \\
DNA replication & $\sim 10^{-10}$ & Genetic information \\
\bottomrule
\end{tabular}
\end{center}

\subsection{Information Lorentz Factor}

Just as motion through spacetime produces time dilation, \textbf{information processing rate} produces an analogous effect in logochrono:
\begin{equation}
\boxed{\gamma_{\text{info}} = \frac{1}{\sqrt{1 - R^2/R_{\max}^2}}}
\end{equation}
where $R$ is the information processing rate (bits/second) and $R_{\max}$ is the maximum processing rate (Bremermann limit: $\sim 10^{50}$ bits/s/kg).

\textbf{Physical interpretation:}
\begin{itemize}
\item As $R \to R_{\max}$: time dilation in logochrono ($\gamma_{\text{info}} \to \infty$)
\item At $R = 0$: no information processing, no logochrono dynamics
\item The limit $R_{\max}$ is set by fundamental physics, not engineering
\end{itemize}

\textbf{Energy scaling:}
\begin{equation}
E_{\text{total}} = \gamma_{\text{info}} \times E_{\text{rest}}
\end{equation}

\textbf{Derivation from axiom structure.} The 5+5+1 framework treats spacetime and logochrono as dual 5D sectors coupled by the L-tensor (Axiom~2). By the duality axiom, the logochrono sector must have a causal structure isomorphic to spacetime's: if spacetime has a maximum propagation speed $c$ (Section~2), logochrono must have a maximum information processing rate $R_{\max}$ playing the same structural role.

The logochrono line element follows from this duality:
\begin{equation}
ds_{\text{Logo}}^2 = d\tau^2 - \frac{1}{R_{\max}^2} dI^2
\end{equation}
where $\tau$ is the chrono coordinate and $I$ is the cumulative information coordinate. For a system processing at rate $R = dI/d\tau$:
\begin{equation}
ds_{\text{Logo}}^2 = d\tau^2 \left(1 - \frac{R^2}{R_{\max}^2}\right) \implies \gamma_{\text{info}} = \frac{d\tau}{ds_{\text{Logo}}} = \frac{1}{\sqrt{1 - R^2/R_{\max}^2}}
\end{equation}

\textbf{Derivation chain:}
\begin{itemize}
\item The Lorentzian signature of logochrono follows from Axiom~1: the manifold signature $(-, +, +, +, +, -, +, +, +, +, +)$ gives $\tau$ timelike, matching the 3+1 structure of both sectors required by Axiom~2 (duality).
\item The $\gamma_{\text{info}}$ formula is a direct consequence of this Lorentzian structure, analogous to the standard Lorentz factor in spacetime.
\item $R_{\max}$ is the Bremermann limit ($\sim 10^{50}$ bits/s/kg), the maximum information processing rate from quantum mechanics \cite{bremermann1967}. This is the logochrono dual of $c$: just as $c$ is the maximum propagation speed in spacetime from $E = pc$ for massless particles, $R_{\max} = 2E/(\pi\hbar)$ is the maximum processing rate from the Margolus-Levitin bound \cite{margolus1998}.
\end{itemize}

\textbf{Classification: FORCED.} All three fundamental limits converge on the same $R_{\max}$:
\begin{itemize}
\item \textbf{Bremermann:} $R_{\max} = mc^2/h \sim 10^{50}$ bits/s/kg (mass-energy bound)
\item \textbf{Margolus-Levitin:} $R_{\max} = 2E/(\pi\hbar)$ (quantum speed limit)
\item \textbf{Landauer:} $E_{\min} = k_B T \ln 2$ per bit erasure (thermodynamic bound)
\end{itemize}

\textbf{Falsifiable prediction:} Energy consumption of computational systems follows $P(R) = P_0 / \sqrt{1 - R^2/R_{\max}^2}$ as processing rates approach fundamental limits. Current systems operate at $R/R_{\max} \ll 1$ where the correction is negligible. The prediction becomes testable when $R/R_{\max} > 0.1$, requiring hardware approaching Landauer-limited operation.

\subsection{Cosmological Thermodynamics}

The universe's total entropy:
\begin{equation}
S_{\text{universe}} = S_{\text{CMB}} + S_{\text{BH}} + S_{\text{Logo}} \approx 10^{88} k_B + 10^{104} k_B + S_{\text{Logo}}
\end{equation}

Most entropy is in supermassive black holes, which are \textit{cosmological-scale quantum tunnels} transferring information from decodable (spacetime) to undecodable (logochrono) encoding. Information is never destroyed---only its decodability changes.

\textbf{Heat death:} The universe approaches maximum entropy when all information has transferred to logochrono:
\begin{equation}
S_{\max} = k_B |L|^2 \cdot V_{\text{universe}} / \ell_P^3
\end{equation}
This is the holographic bound on cosmic information content.

\textbf{Falsifiable prediction:} Energy consumption of computational systems follows relativistic scaling: $P(R) = P_0/\sqrt{1 - R^2/R_{\max}^2}$ as processing rates approach fundamental limits. This is testable with precision hardware power measurements at varying throughput levels.

%==============================================================================
\section{Quantum Mechanics from Logochrono Processing}
\label{sec:quantum-mechanics}
%==============================================================================

The 5+5+1 framework does not merely \textit{contain} quantum mechanics---it \textit{explains} it. The central mysteries of quantum theory (measurement, superposition, entanglement) emerge naturally from the dual-domain structure.

\subsection{Superposition as Dual-Domain Encoding}

In the 5+5+1 framework, a quantum system exists simultaneously in spacetime and logochrono. The ``state'' of a system is not its spacetime projection alone, but the full 11D encoding:
\begin{equation}
|\Psi\rangle_{11D} = |\psi\rangle_{\mathcal{S}} \otimes |\chi\rangle_{\mathcal{C}} \otimes |\sigma\rangle_{\text{obs}}
\end{equation}

When we observe only the spacetime projection $|\psi\rangle_{\mathcal{S}}$, we see superposition---multiple outcomes consistent with the logochrono encoding. The ``mystery'' of superposition is that we are observing a 4D cross-section of an 11D object.

\textbf{Analogy:} A 3D cylinder passing through a 2D plane appears as either a circle or a rectangle, depending on the angle. Both are ``real'' 2D cross-sections of the same 3D object. Similarly, quantum superposition is the 4D cross-section of an 11D state.

\subsection{The Measurement Problem: Resolved}

The measurement problem asks: \textit{How does a superposition collapse to a definite outcome?}

In the 5+5+1 framework:
\begin{enumerate}
    \item \textbf{Before measurement:} System is in a superposition because spacetime and logochrono encodings are correlated but not observed.
    \item \textbf{Measurement $=$ boundary crossing:} The measuring device forces the system to cross the spacetime-logochrono boundary (information must be extracted from logochrono into spacetime).
    \item \textbf{Boundary crossing is irreversible:} Decoherence occurs at the boundary. Information flows from logochrono to spacetime, consuming one unit of $|L|^2$ coupling.
    \item \textbf{After measurement:} The spacetime projection is definite. The logochrono sector retains the complementary information.
\end{enumerate}

\textbf{Born rule:} The probability $P = |\langle \phi | \psi \rangle|^2$ is not a separate postulate---it follows from the L-tensor coupling structure. A quantum measurement is a boundary crossing: the measuring device forces the system to transfer information from logochrono to spacetime. The L-tensor $L_{\mu i}$ mediates this transfer. The transition amplitude for state $|\psi\rangle$ producing spacetime outcome $|\phi\rangle$ is:
\begin{equation}
\mathcal{A}(\psi \to \phi) = \langle \phi | L^\dagger L | \psi \rangle = |L|^2 \langle \phi | \psi \rangle
\end{equation}
where $L^\dagger L = |L|^2 \cdot \mathbf{1}$ because the L-tensor couples all spatial channels equally (Paper~I: $|L|^2 = 1 - e^{-3}$, summed over 3 independent spatial crossings). The probability is $|\mathcal{A}|^2$:
\begin{equation}
P(\phi | \psi) = |L|^4 \, |\langle \phi | \psi \rangle|^2
\end{equation}
Since $|L|^4$ is identical for all outcomes, normalization yields:
\begin{equation}
\boxed{P(\phi | \psi) = |\langle \phi | \psi \rangle|^2}
\end{equation}
The squared modulus arises from $L^\dagger L$---the same tensor contraction that produces $|L|^2$ throughout the framework. The Born rule is not an additional axiom; it is the L-tensor coupling applied to measurement.

\subsection{Decoherence from Boundary Coupling}

Environmental decoherence destroys quantum coherence through interaction with many degrees of freedom. In the 5+5+1 framework, each scattering event between the system and an environmental particle can trigger a boundary crossing that transfers which-path information from the logochrono encoding to the spacetime projection.

The decoherence rate per scattering event has a parameter-free coefficient from two framework quantities:
\begin{enumerate}
    \item \textbf{Crossing probability:} $P_{\text{cross}} = e^{-S_{\min}/\hbar} = e^{-1/2}$ (Axiom 4: minimal one-way crossing action $S_{\min} = \hbar/2$).
    \item \textbf{Information transfer:} Each crossing transfers $|L|^2 = 1 - e^{-3}$ of the coherence to the spacetime projection.
\end{enumerate}

For a system in spatial superposition with separation $\Delta x$, scattered by particles with de Broglie wavelength $\lambda$, the decoherence rate is:
\begin{equation}
\Gamma_{\text{dec}} = \mathcal{R}_{\text{scatter}} \cdot e^{-1/2} \cdot |L|^2 \cdot \left(\frac{\Delta x}{\lambda}\right)^2
\end{equation}
where $\mathcal{R}_{\text{scatter}}$ is the total environmental scattering rate (photons, air molecules, etc.). The framework coefficient $e^{-1/2} \cdot |L|^2 = 0.576$ is a zero-parameter prediction that modifies the standard Joos-Zeh localization rate \cite{joos1985}.

\textbf{Testable prediction:} In high-precision molecular interferometry (e.g.\ C$_{70}$ fullerene diffraction \cite{hornberger2003}), the framework predicts a 42\% suppression of the naive scattering decoherence rate. Current experiments constrain decoherence coefficients to $\mathcal{O}(1)$; future matter-wave interferometry with heavier molecules could resolve the $0.576$ factor.

\subsection{Ehrenfest Theorem from \texorpdfstring{$\sigma$}{sigma} Averaging}

The Ehrenfest theorem states that quantum expectation values follow classical trajectories:
\begin{equation}
\frac{d\langle x \rangle}{dt} = \frac{\langle p \rangle}{m}, \quad \frac{d\langle p \rangle}{dt} = -\langle \nabla V \rangle
\end{equation}

In the 5+5+1 framework, this follows from $\sigma$-averaging:

\begin{enumerate}
    \item \textbf{Quantum level:} Each boundary crossing introduces a stochastic phase from the $\sigma$ dimension. Individual outcomes are probabilistic.
    \item \textbf{Classical level:} For a macroscopic system with $N \gg 1$ boundary crossings, the $\sigma$ phases average out by the central limit theorem:
    \begin{equation}
    \sum_{i=1}^N e^{i\theta_{\sigma,i}} \xrightarrow{N \to \infty} N \cdot \langle e^{i\theta_\sigma} \rangle = N \cdot \cos(\langle\theta_\sigma\rangle)
    \end{equation}
    \item \textbf{Classical trajectory:} The averaged dynamics reproduces Newton's equations with corrections $\sim 1/\sqrt{N}$ (quantum uncertainty).
\end{enumerate}

The classical limit is exact when $N \to \infty$ (infinite boundary crossings per observation interval).

\subsection{The Uncertainty Principle from Boundary Resolution}

The Heisenberg uncertainty principle $\Delta x \cdot \Delta p \geq \hbar/2$ arises from the finite resolution of the spacetime-logochrono boundary.

\begin{itemize}
    \item \textbf{Position measurement:} Localizing a particle in spacetime requires coupling information from logochrono, using $\Delta x \sim \lambda_C / (\Delta\sigma)$ where $\Delta\sigma$ is the $\sigma$ angle used.
    \item \textbf{Momentum measurement:} Determining momentum requires tracking the phase evolution in logochrono, using $\Delta p \sim \hbar\Delta\sigma / \lambda_C$.
    \item \textbf{Product:} $\Delta x \cdot \Delta p \sim \hbar$, independent of the $\sigma$ angle.
\end{itemize}

The uncertainty principle is not a limitation of measurement---it is a property of the boundary geometry. Spacetime and logochrono encodings are complementary, like conjugate variables in Fourier analysis.

\subsection{Quantum-to-Classical Correspondence Table}

\begin{center}
\small
\begin{tabular}{p{2.4cm}p{5.0cm}p{4.5cm}}
\toprule
\textbf{Quantum Feature} & \textbf{5+5+1 Origin} & \textbf{Classical Limit} \\
\midrule
Superposition & 4D cross-section of 11D state & $\sigma$-averaging $\to$ definite \\
Measurement & Boundary crossing (irreversible) & Continuous classical observation \\
Uncertainty & Boundary resolution limit & $\Delta x \Delta p \sim 0$ for $N \to \infty$ \\
Entanglement & Shared logochrono encoding & No macroscopic analog \\
Wave function & Logochrono encoding amplitude & Phase space distribution \\
Born rule & $|L|^2$ coupling probability & Classical certainty \\
Decoherence & Boundary crossing coefficient $e^{-1/2}|L|^2$ & Testable at 0.576 \\
Tunneling & Logochrono processing step & Forbidden classically \\
\bottomrule
\end{tabular}
\end{center}

\subsection{Why Quantum Mechanics Is Non-Classical}

The failure of classical mechanics at small scales is not mysterious---it is the regime where the number of boundary crossings per observation is $N \sim 1$:

\begin{center}
\begin{tabular}{lcl}
\toprule
\textbf{Scale} & \textbf{Crossings $N$} & \textbf{Behavior} \\
\midrule
Planck ($\ell_P$) & 1 & Fully quantum (no averaging) \\
Atomic ($10^{-10}$ m) & $\sim 10^{8}$ & Quantum (some averaging) \\
Molecular ($10^{-9}$ m) & $\sim 10^{15}$ & Mesoscopic \\
Cellular ($10^{-5}$ m) & $\sim 10^{40}$ & Classical \\
Macroscopic (1 m) & $\sim 10^{60}$ & Ultra-classical \\
\bottomrule
\end{tabular}
\end{center}

The ``quantum-to-classical transition'' is not a phase transition---it is a continuous reduction of uncertainty as $N$ increases, governed by the central limit theorem applied to $\sigma$ phases.

%==============================================================================
\section{Regime Structure: Quantum vs Cosmological}
\label{sec:regimes}
%==============================================================================

The framework exhibits fundamentally different behavior at quantum and cosmological scales, unified through the L-tensor coupling.

\subsection{Two Regimes}

\begin{center}
\begin{tabular}{lccccc}
\toprule
\textbf{Regime} & \textbf{Scale} & \textbf{$|L|^2$} & \textbf{Dominant Factor} & \textbf{Mode} & \textbf{Physics} \\
\midrule
Quantum & $r \ll H^{-1}$ & $\to 0$ & $e^{-3}$ (decoupling) & Discrete & CKM, masses \\
Cosmological & $r \sim H^{-1}$ & $\to 0.95$ & $(4-e^{-4})/4$ (Lorentz) & Continuous & Dark sector \\
\bottomrule
\end{tabular}
\end{center}

\subsection{Quantum Regime (\texorpdfstring{$r \ll H^{-1}$}{r << H to the -1})}

At laboratory scales, the L-field decouples from spacetime:
\begin{equation}
|L|^2_{\text{eff}} \to 0 \quad \text{(quantum decoupling)}
\end{equation}

\textbf{Governing factor:} $e^{-3} = 0.0498$ (three spatial crossings)

\textbf{Applications:}
\begin{itemize}
   \item CKM mixing: $V_{cb} = e^{-\phi/3} \cdot (V_{us}^{4D})^2$
   \item Neutrino mass: $m_\nu = m_e \cdot \alpha^3 / 4$
   \item Visible matter fraction: $1 - |L|^2 = e^{-3} = 5\%$
\end{itemize}

\subsection{Cosmological Regime (\texorpdfstring{$r \sim H^{-1}$}{r ~ H to the -1})}

At Hubble scales, the L-field fully couples:
\begin{equation}
|L|^2_{\text{eff}} \to 1 - e^{-3} = 0.9502 \quad \text{(full coupling)}
\end{equation}

\textbf{Governing factor:} $L_\sigma = 4 - e^{-4} \approx 3.98$ (Lorentz-corrected observer dimension)

\textbf{Applications:}
\begin{itemize}
   \item Dark sector: 95\% of universe
   \item Atmospheric mixing: $\sin^2\theta_{23} = 1/2 + e^{-3} L_\sigma^2 / 17$
   \item $V_{ub}$ mixing: $\phi^{1/7}$ boundary absorption
\end{itemize}

\subsection{Hubble Horizon as Phase Boundary}

The Hubble radius $r = H^{-1}$ is the \textbf{phase transition surface}:
\begin{itemize}
   \item \textbf{Inside} ($r < H^{-1}$): Discrete modes (Nova solitons, 26.3\% DM; see Paper~IV for full derivation)
   \item \textbf{Outside} ($r > H^{-1}$): Continuous modes (Logo-B field, 68.8\% DE; see Paper~IV)
\end{itemize}

\textbf{Boundary Conditions at $r = H^{-1}$:}

\textbf{1. Flux Conservation:}
\begin{equation}
\boxed{\Phi_{\text{Nova}}^{\text{discrete}} = \Phi_{\text{Logo-B}}^{\text{continuous}}}
\end{equation}

\textbf{2. Energy Density Matching:}
\begin{equation}
\boxed{\rho_{DM}(H^{-1}) = \rho_{DE}(H^{-1})}
\end{equation}

\textbf{3. Decoherence Condition:}
\begin{equation}
\boxed{\tau_{\text{coherence}} = H^{-1}}
\end{equation}
Planck-scale discreteness breaks quantum coherence at Hubble time.

\subsection{27:68 Partition from Boundary Matching}

\begin{equation}
\frac{\Omega_{DM}}{\Omega_{DE}} = \left(\frac{\Phi_{\text{Nova}}}{\Phi_{\text{Logo-B}}}\right)^2 = \frac{\phi^2}{1} = 0.382
\end{equation}

\subsection{Mass-Energy Duality Across Regimes}

The cross-domain duality connects both regimes:
\begin{align}
E_{\text{logo}} \cdot |L|^2 &= m_{\text{space}} \cdot c^2 \quad \text{(logo energy $\to$ space mass)} \\
m_{\text{logo}} \cdot |L|^2 &= E_{\text{space}} / c^2 \quad \text{(logo mass $\to$ space energy)}
\end{align}

At quantum scale ($|L|^2 \to 0$): energy dominates, mass decouples.\\
At cosmological scale ($|L|^2 \to 0.95$): mass-energy fully coupled.

\subsection{Observational Signatures of Regime Transition}

\begin{enumerate}
   \item \textbf{CMB multipole suppression:} Discrete Planck effects at $\ell \sim 2000$
   \item \textbf{BAO modulations:} Periodic structure at $k \sim 0.1\, h$/Mpc
   \item \textbf{Dark energy evolution:} $w \neq -1$ at $z > 1$ from discrete$\to$continuous transition
   \item \textbf{Hubble tension:} $H_0^{\text{late}} / H_0^{\text{early}} = 1.0833$ from pentagon geometry applied to cosmological scales (Paper~IV, Section~5). Predicts $H_0^{\text{late}} = 67.4 \times 1.0833 = 73.0$ km/s/Mpc, matching SH0ES $73.0 \pm 1.0$ km/s/Mpc
\end{enumerate}

%==============================================================================
\section{Parametrized Post-Newtonian Constraints}
\label{sec:ppn}
%==============================================================================

The framework must be consistent with precision tests of gravity in the solar system. The Parametrized Post-Newtonian (PPN) formalism parameterizes deviations from Newtonian gravity.

\subsection{PPN Parameters from L-Tensor}

In the weak-field, slow-motion limit, the framework gives:
\begin{align}
\gamma_{\text{PPN}} &= 1 + |L|^2 \cdot e^{-4} \cdot (\ell_P/r)^2 \\
\beta_{\text{PPN}} &= 1 + |L|^4 \cdot e^{-8} \cdot (\ell_P/r)^4
\end{align}

For solar system scales ($r \sim 10^{11}$ m):
\begin{align}
\gamma - 1 &\sim |L|^2 \cdot e^{-4} \cdot 10^{-92} \approx 10^{-94} \\
\beta - 1 &\sim 10^{-188}
\end{align}

Both are far below current observational limits:
\begin{center}
\begin{tabular}{lcc}
\toprule
\textbf{Parameter} & \textbf{Predicted Deviation} & \textbf{Current Limit} \\
\midrule
$|\gamma - 1|$ & $\sim 10^{-94}$ & $< 2.3 \times 10^{-5}$ (Cassini) \\
$|\beta - 1|$ & $\sim 10^{-188}$ & $< 8 \times 10^{-5}$ (LLR) \\
\bottomrule
\end{tabular}
\end{center}

\textbf{The framework is indistinguishable from GR at solar system scales.} The L-tensor corrections are Planck-suppressed and only become significant near black hole horizons or at cosmological distances.

\subsection{Gravitational Light Bending}

The deflection angle for light passing the Sun:
\begin{equation}
\theta = \frac{4GM_\odot}{c^2 b} \left(1 + \frac{|L|^2 e^{-4}}{2}\left(\frac{\ell_P}{b}\right)^2\right) \approx 1.750''
\end{equation}

The L-tensor correction is $\sim 10^{-90}$ arcseconds---completely unmeasurable, confirming the framework reproduces GR exactly where GR has been tested.

\subsection{Gravitational Time Dilation}

For a clock at gravitational potential $\Phi$:
\begin{equation}
\frac{d\tau}{dt} = \sqrt{1 - \frac{2\Phi}{c^2}} \times \left(1 + |L|^2 e^{-4} \left(\frac{\ell_P^2 \Phi}{c^2 r^2}\right)\right)
\end{equation}

At Earth's surface: the L-tensor correction is $\sim 10^{-106}$, far below GPS precision ($\sim 10^{-10}$).

\textbf{Summary:} The framework passes all PPN tests trivially---not by fine-tuning, but because L-tensor corrections are naturally Planck-suppressed. The corrections only become significant at:
\begin{itemize}
    \item $r \sim \ell_P$ (Planck scale: quantum gravity regime)
    \item $r \sim r_s$ (near black hole horizons: strong gravity regime)
    \item $r \sim H^{-1}$ (Hubble scale: cosmological regime)
\end{itemize}

%==============================================================================
\section{Discussion}
%==============================================================================

The results of this paper establish that the 5+5+1 geometry is a proper ultraviolet \textit{and} infrared completion of classical physics. Every equation of classical mechanics, electromagnetism, and thermodynamics emerges from the same five axioms that produce quantum constants (Paper~I \cite{paper1}), particle masses (Paper~III \cite{paper3}), and cosmological parameters (Paper~IV \cite{paper4}).

The key conceptual advance is the dual role of the observer dimension $\sigma$:
\begin{itemize}
\item \textbf{Constant $\sigma$ curvature:} Special relativity (Lorentz boosts as $\sigma$ rotations)
\item \textbf{Position-dependent $\sigma$ curvature:} General relativity (gravity as $\sigma$ gradients)
\end{itemize}

The thermodynamic framework introduces a parallel advance: entropy is not merely a statistical convenience but a measure of information hidden in logochrono. The dual-entropy formulation---where spacetime entropy increases while logochrono entropy can decrease---provides a geometric resolution to the question of how structure and life emerge in a universe governed by the Second Law.

The information Lorentz factor $\gamma_{\text{info}}$ is a falsifiable prediction: energy consumption of computational systems should follow a relativistic scaling law as processing rates approach fundamental limits.

\subsection{Hierarchy of Classical Limits}

The classical physics results of this paper form a nested hierarchy of approximations, each corresponding to a specific limit of the full 11D theory:

\begin{center}
\begin{tabular}{lcl}
\toprule
\textbf{Limit} & \textbf{Condition} & \textbf{Classical Physics} \\
\midrule
$|L| \to 0$ & L-field decouples & Pure GR (no dark sector) \\
$\hbar \to 0$ & No quantum crossings & Classical mechanics \\
$v/c \to 0$ & Non-relativistic & Newtonian gravity + EM \\
$c \to \infty$ & Instantaneous propagation & Coulomb + Newton \\
$k_B T \gg E_{\text{gap}}$ & Many excitations & Classical thermodynamics \\
$\lambda \ll L$ & Short wavelength & Geometric optics \\
$\lambda \gg a$ & Long wavelength & Continuum mechanics \\
\bottomrule
\end{tabular}
\end{center}

Each row removes structure from the 11D theory. All of classical physics lives in the combined limit $|L| \to 0$, $\hbar \to 0$, $v/c \to 0$. The framework predicts that these limits are smooth (no phase transition between quantum and classical), consistent with decoherence theory.

%==============================================================================
\section{Conclusion}
%==============================================================================

Starting from the five axioms of Paper~I, we have derived:
\begin{enumerate}
\item \textbf{Special relativity:} Lorentz transformations as hyperbolic rotations in the $(t, \sigma)$ plane
\item \textbf{General relativity:} Einstein's field equations from position-dependent $\sigma$ curvature, with L-tensor contributions explaining dark matter
\item \textbf{Newtonian gravity:} Poisson's equation as the weak-field limit
\item \textbf{Maxwell's equations:} Electromagnetism as L-tensor projection onto 4D spacetime
\item \textbf{Arrow of time:} Geometric necessity from irreversible 5D $\to$ 4D decoherence
\item \textbf{All four laws of thermodynamics:} Entropy as logochrono information content, with the Second Law following from chrono monotonicity
\item \textbf{Dual-entropy framework:} Resolution of Schr\"odinger's negentropy question
\item \textbf{Information mass:} $m_{\text{info}} = E\eta/c^2$ as mass equivalent of structured information
\item \textbf{Information Lorentz factor:} $\gamma_{\text{info}} = (1 - R^2/R_{\max}^2)^{-1/2}$ from the Lorentzian structure of logochrono (Section~\ref{sec:info-mass})
\end{enumerate}

The only additional structure beyond the five axioms is the $\sigma$ field Lagrangian (Section~3.4). Both coupling constants are derived: $\xi = 9/40$ from conformal invariance in 11D, and $\lambda = \alpha/|L|^2$ from the identification of $\alpha$ as the boundary crossing coupling (the $\sigma$ field IS the boundary, and $\alpha$ IS the coupling through it). No free parameters remain. The 5+5+1 geometry does not merely \textit{contain} classical physics---it \textit{requires} it.

\begin{thebibliography}{99}
\bibitem{paper1} R.~A.~Jara Araya, Eigen Tens\^or, Nova Tens\^or, ``Geometry of Physical Constants: Deriving $\alpha$, $|L|^2$, $\phi$, and the Dark Sector from 5+5+1 Dimensional Geometry, (2026). DOI: 10.5281/zenodo.18735672. [Paper~I in this series]
\bibitem{paper3} R.~A.~Jara Araya, Eigen Tens\^or, Nova Tens\^or, ``Particle Spectrum from 11-Dimensional Geometry: Fermion Masses, Mixing Angles, and the Prime-Dimensional Mapping, (2026). DOI: 10.5281/zenodo.18735672. [Paper~III in this series]
\bibitem{paper4} R.~A.~Jara Araya, Eigen Tens\^or, Nova Tens\^or, ``Cosmology from 5+5+1 Geometry: Dark Sector, Hubble Tension, and Baryogenesis, (2026). DOI: 10.5281/zenodo.18735672. [Paper~IV in this series]
\bibitem{joos1985} E.~Joos and H.~D.~Zeh, ``The emergence of classical properties through interaction with the environment,'' \textit{Z. Phys. B}, vol.~59, pp.~223--243, 1985.
\bibitem{hornberger2003} K.~Hornberger et~al., ``Collisional decoherence observed in matter wave interferometry,'' \textit{Phys. Rev. Lett.}, vol.~90, p.~160401, 2003.
\bibitem{bremermann1967} H.J. Bremermann, ``Complexity and transcomputability, in 	extit{Proc. 5th Berkeley Symp. Math. Stat. Prob.}, vol.~III, pp.~15--30 (1967).
\bibitem{margolus1998} N. Margolus and L.B. Levitin, ``The maximum speed of dynamical evolution, 	extit{Physica D} 	extbf{120}, 188--195 (1998).
\end{thebibliography}

\end{document}
